\documentclass[11pt]{article}
\usepackage{mathpazo}
\usepackage{url}
\usepackage{verbatim}

\usepackage[T1]{fontenc}
\usepackage{textcomp}
\usepackage[scaled]{luximono}
\usepackage{latexsym}
\usepackage{listings}
\usepackage{ps}


\textwidth=15cm
\textheight=22cm
\topmargin=0pt
\headheight=0pt
\oddsidemargin=1cm
\headsep=0pt
\renewcommand{\baselinestretch}{1.1}
\setlength{\parskip}{0.20\baselineskip plus 1pt minus 1pt}
\parindent=0pt

\title{\textbf{Principles of the Spin Model Checker\\
Supplementary Material on Spin Version 6}}

\author{\textbf{Mordechai (Moti) Ben-Ari}\\
\url{http://stwww.weizmann.ac.il/g-cs/benari/}}
%\date{}
\begin{document}
\maketitle
\thispagestyle{empty}

\vfill

\begin{center}
Copyright 2010 by Mordechai (Moti) Ben-Ari.
\end{center}

This work is licensed under the Creative Commons Attribution
Non-Commercial No Derivs 3.0 License. To view a copy of this license,
visit \url{http://creativecommons.org/licenses/by-nc-nd/3.0/}; or, (b)
send a letter to Creative Commons, 543 Howard Street, 5th Floor, San
Francisco, California, 94105, USA.

\newpage

In December 2010, Version 6 of \spn{} was released; this version
introduced new structures into the modeling language \prm{}. This
document describes the new structures and shows how they can be used in
the examples from \textit{Principles of the Spin Model Checker}. The
latest software archive \p{ps-programs.zip} contains these modified
programs.

The new version of \spn{} also made changes to the format of its output
data. This required some modifications of the \jspin{} development
environment; the new version can be downloaded from
\url{http://code.google.com/p/jspin/}.

The modifications of \prm{} include: a \p{for}-statement that replaces
the use of a macro; a \p{select}-statement for nondeterministic
selection of values; the scope rules have been changed; formulas in
linear temporal logic (LTL) can be included within the \prm{} source;
LTL formulas can contain expressions; LTL formulas can use keywords for
the temporal operators.

Version 6 also supports multiple \p{never}-claims and synchronous
products of claims but this will not be discussed here.

The notation \S{}n.m refers to \textbf{Listing~n.m} in
\textit{Principles of the Spin Model Checker}.

\section{The \p{for}-statement}

\prm{} uses the nondeterministic guarded command \p{do} to support
loops. When you want a simple counting loop, the following code can used
(\S{}1.8):

\begin{frag}
  byte i; 
  i = 1; 
  do 
  :: i > 10 -> break 
  :: else  ->
       /* body of the loop */
       i++
  od;
\end{frag}
This code is frequently encapsulated in a pair of macros: \texttt{for}
and \texttt{rof}.

A new control structure implements a counting loop directly:  
\begin{frag}
  byte i; 
  for (i : 1..10) {
    /* body of the loop */
  }
\end{frag}

This is equivalent to the following code which is different from that
used in \S{}1.8:

\newpage

\begin{frag}
  byte i; 
  i = 1; 
  do 
  :: i<= 10 ->
       /* body of the loop */
       i++
  :: else  -> break
  od;
\end{frag}


The bounds can be expressions:
\begin{frag}
  for (i : 0..N-1)
\end{frag}

\prm{} differs from many programming languages in that the index
variable must be explicitly declared prior to the loop. 

Many of the examples (\S{}1.8, \S{}3.5, \S{}6.2, \S{}11.14)
were changed to use the \p{for}-statement.

\section{The \p{select}-statement}

To choose a value nondeterministically, you can use an \p{if}-statement
with true (absent) guards. For example, to choose the next row in the
8-queens problem, we used (encapsulated within an \p{inline} construct
\p{Choose} in \S{}11.5):

\begin{frag}
	if
	:: row = 1
	:: row = 2
	   ...
	:: row = 8
	fi
\end{frag}

This can now be succinctly written as:
\begin{frag}
  select(row : 1..8)
\end{frag}
The bounds can be expressions and the variable must be previously defined.

Since \p{true} and \p{false} are just names for the constants 1 and 0,
the following is correct:
\begin{frag}
  bool b;
  select(b : false..true)
\end{frag}
as a replacement for:
\begin{frag}
  bool b;
  if :: b = false :: b = true fi
\end{frag}

The implementation of \p{select} uses a nondeterministic loop
(Section~4.6.2):
\begin{frag}
  row = 1;
  do
  :: row < 8 -> row++
  :: break
  od
\end{frag}

\section{Scope rules}

Previously, \p{inline} definitions did not create a new scope for
variable declarations, although the syntax is implies that it
does. Given the definition:
\begin{frag}
inline write(n) {
  byte nsq;
  nsq = n*n;
  printf("n = %d, n squared = %d\n", n, nsq)  
}
\end{frag}

The following is illegal:
\begin{frag}
active proctype P() {
  byte a = 10, b = 12;
  write(a);
  write(b)
}
\end{frag}

because the variable \p{nsq} is redeclared by \p{write(b)} within the
scope of the \p{proctype}. \spn{}, however, runs the program as
expected, although an error message is given. Now, the program runs
without the error message because the scope of \p{nsq} is limited to the
\p{inline} constructs.

\warn{no dynamic scope}{\emph{\bfseries Do not} write \p{byte nsq =
n*n;} in the \p{inline} definition!\\The scope rule refers only to the
\emph{name} itself; all \emph{variables} within a \p{proctype} are still
collected and placed at the beginning of the \p{proctype} and
initialized \emph{once} when the \p{proctype} is activated.}

Note: \S{}6.3 was and is illegal because an array name cannot be passed
to an \p{inline} definition, although this is not expressly forbidden in
the \prm{} reference manual.

\newpage

\section{LTL formulas}

Consider an algorithm for solving the critical-section problem:
\begin{frag}
bool csp = false, csq = false;

active proctype P() {
  do
  ::
    ...
    csp = true;   /* Enter critical section */
    csp = false; /* Leave critical section */
    ...
  od
}

active proctype Q() { /* Similar */ }
\end{frag}
To verify mutual exclusion we have to show that the following LTL
formula holds:
\begin{frag}
[]!(csp && csq)
\end{frag}
while to verify absence of starvation the formula is:
\begin{frag}
[]<>csp
\end{frag}

In \spn{} this is done by: (a) writing the LTL formula in a file (or
an argument); (b) negating the formula; (c) translating it into a
\p{never}-claim; (d) running a verification with the \p{never}-claim. In
previous versions of \spn{}, the user had to negate the formula and then
run a command to translate it into a \p{never}-claim; the
\p{never}-claim was then included as an argument to the
verification.\footnote{An environment can simplify the process. For
example, \jspin{} automatically negates the LTL formula, translates it
into a \p{never}-claim when a button is clicked and then includes the
claim in the arguments to the verification.} Currently, the LTL formula
can be written \emph{within} the \prm{} program:

\begin{frag}
bool csp = false, csq = false;

ltl { []!(csp && csq) }
\end{frag}
\spn{} will \emph{automatically} negate the formula and translate it
into a \p{never}-claim when the verification is performed.

To simplify matters even further, several \emph{named} LTL formulas can
be included and the choice of the formula to use is made when the
verification is done:

\begin{frag}
ltl mutex { []!(csp && csq) }
ltl nostarvation { []<>csp }
\end{frag}

When there is only one LTL formula in a program, just perform the
verification (\verb+spin -a+, \verb+gcc+, \verb+pan+) and the formula
will be automatically used by \spn{}. When there is more than one
(named) LTL formula, you must specify which formula is to be used in a
verification.

\begin{runj}{selecting an LTL formula}
Enter one \emph{name} in the text field labeled \p{LTL formula} and
select \p{LTL name} from the toolbar or the \p{Convert} menu.
\end{runj}

\begin{runcl}{selecting an LTL formula}
Specify one \emph{name} when the verification is done:
\begin{frag}
pan -N mutex
\end{frag}
\end{runcl}

The inclusion of multiple LTL formulas within a \prm{} program
simplifies the configuration management of a project because all the
correctness properties are contained within the same file as the source
code of the model.

Additional new features concerning LTL formulas:
\begin{itemize}
\item Expressions can be used in an internal LTL formula, so it is no
longer necessary to define symbols for this purpose:
\p{ltl \{ [](critical <= 1) \}}.
\item A remote reference is considered an expression, so
\p{ltl \{ []!(P@cs && Q@cs) \}} can be given 
as the correctness specification of \S{}5.3 without defining the symbol
\p{mutex}. The formula must appear \emph{after} the
\p{proctype}'s where the labels are defined.
\item 
Temporal operators can be given as keywords:
\begin{frag}
ltl mutex { always !(csp && csq) }
ltl nostarvation { always eventually csp }
\end{frag}
\end{itemize}


\end{document}
